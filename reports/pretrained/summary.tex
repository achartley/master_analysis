\documentclass[12pt, notitlepage]{article}
%\usepackage[backend=biber]{biblatex}
\usepackage{amsmath}
\usepackage{listings}
\usepackage{graphicx}
\usepackage{caption}
%\usepackage{subcaption}
\usepackage{commath}
\usepackage{hyperref}
\usepackage{url}
\usepackage{xcolor}
\usepackage{textcomp}
\usepackage{dirtytalk}
\usepackage{listings}
\usepackage{wasysym}
\usepackage{float}
\usepackage{listings}
\usepackage[linesnumbered,lined,boxed,commentsnumbered]{algorithm2e}
\usepackage{subfig}

% Packages from derivations_fullproblem.tex
\usepackage[squaren]{SIunits}
\usepackage{a4wide}
\usepackage{array}
\usepackage{cancel}
\usepackage{amsmath}
\usepackage{amsfonts}
\usepackage{amssymb}
\usepackage{graphicx}
\usepackage{enumerate}
\usepackage{titling}

% Parameters for displaying code.
\lstset{language=python}
\lstset{basicstyle=\ttfamily\small}
\lstset{frame=single}
\lstset{keywordstyle=\color{red}\bfseries}
\lstset{commentstyle=\itshape\color{blue}}
\lstset{showspaces=false}
\lstset{showstringspaces=false}
\lstset{showtabs=false}
\lstset{breaklines}

% Define new commands
\newcommand{\expect}[1]{\langle #1 \rangle}
\newcommand{\dd}[1]{\ \text{d}#1}
\newcommand{\f}[2]{\frac{#1}{#2}} 
\newcommand{\tf}[2]{\tfrac{#1}{#2}}
\newcommand{\beq}{\begin{equation*}}
\newcommand{\eeq}{\end{equation*}}
\newcommand{\bra}[1]{\langle #1|}
\newcommand{\ket}[1]{|#1 \rangle}
\newcommand{\braket}[2]{\langle #1 | #2 \rangle}
\newcommand{\av}[1]{\left| #1 \right|}
\newcommand{\op}[1]{\widehat{#1}}
\newcommand{\braopket}[3]{\langle #1 | \op{#2} | #3 \rangle}
\newcommand{\ketbra}[2]{\ket{#1}\bra{#2}}
\newcommand{\pp}[1]{\frac{\partial}{\partial #1}}
\newcommand{\ppn}[1]{\frac{\partial^2}{\partial #1^2}}
\newcommand{\up}{\left|\uparrow\rangle\right.}
\newcommand{\down}{\left|\downarrow\rangle\right.}
\newcommand{\updown}{\left|\uparrow\downarrow\rangle\right.}
\newcommand{\downup}{\left|\downarrow\uparrow\rangle\right.}
\newcommand{\ua}{\uparrow}
\newcommand{\da}{\downarrow}
% Add bibliography
\begin{document}


\title{Summary - Classifying Events in Scintillator Data with Pre-Trained Neural Networks}
\author{Geir Tore Ulvik}
\maketitle
\section{Introduction}
A short summary of current work on using pre-trained neural networks to classify
single and double events in scintillator data.

\section{What are pre-trained networks and how do we use them?}
As the name suggests a pre-trained network is a network that has already
been trained, usually on very large amounts of data over a significant amount
of time. This is the case for the networks explored in this work. The networks
have all been trained on the "ImageNet" database and are on the cutting edge in
computer vision. They have also been analyzed extensively, increasing the amount
of insight in the learning process.

But we're not interested in classifying ImageNet data, we have our own data,
so how can we use the power of these networks for our own purposes?
The answer is \textbf{feature extraction}. When feeding an image forward
through the trained network, the various filters (or "kernels") react to different features
of the image, like edges or lines, general shapes etc. What we end up with on the other side
of the network is a "feature representation" of the input image. Hopefully, an input image
of say a cat and a car will produce significantly different feature representations, allowing
for classification. In our case we desire that the feature representations for single and
double events are different. This is the first thing we test when looking at which networks
might work for us.

\section{Data preparation}
In order to use the pre-trained networks, our data needs to be prepared to fit some requirements.
\subsection{Shape and Size}
The pre-trained networks are trained on RGB color images with a shape of $(224, 224, 3)$
corresponding to $(p_x, p_y, channels)$. Our data is much smaller, $(16, 16, 1)$. 
We solve the number of channels by concatenating our image data to itself, giving us 3 identical channels. 
Additionally, for the networks to accept our data we need to replace the input layer of each network,
with one that accepts our image size.
\subsection{Normalization of pixel values}
To reduce the risk of exploding and/or vanishing gradients, pixel values are commonly
normalized to $[0, 1]$ or to $[-1, 1]$ (with zero-mean). Here we have opted for a min-max scaling to the $[0, 1]$ interval.
The minimum and maximum pixel values are fetched for the entire dataset, such that intensity differences
between images are preserved. Min-max scaling is calculated as
\begin{equation}
    \text{scaled image} = \f{\text{image} - \mu_{image}}{I_{max} - I_{min}},
\end{equation}
where $I_{max}$ and $I_{min}$ refer to the maximum and minimum pixel intensity for the dataset.


\section{Framework}
\begin{itemize}
    \item Python3
    \item TensorFlow (with Keras API)
    \item Scikit-Learn
    \item SciPy
\end{itemize}
\section{Networks}
The following networks have been explored to some degree.
\begin{itemize}
    \item DenseNet\{121, 169, 201\}
    \item InceptionResNetV2
    \item InceptionV3
    \item MobileNet
    \item MobileNetV2
    \item NASNetLarge
    \item NASNetMobile
    \item ResNet50
    \item VGG16
    \item VGG19
    \item Xception
\end{itemize}
\section{Feature Extraction}
To compare the extracted feature representations and find out if they may allow for classification,
a Kolmogorov-Smirnov(KS) two-sample test was performed. The test compares each feature's distribution
for single events and double events. The same test is also performed for single events and double events 
where the relative distance between events is $< 3mm$. The output p-value of the KS two-sample test
indicates whether or not we can reject the null hypothesis. The null hypothesis for one feature in this 
case is that for both single and double events the feature is drawn from the same underlying distribution.
For p-values lower than 0.01 we can generally reject the null-hypothesis. 

However, keep in mind that even if the feature distributions are different it's not a guarantee that 
classification will work, but it may indicate which networks won't perform well.
\subsection{Results}
The tables \ref{tab:features} and \ref{tab:features-close} show the number of features for each network,
and the ratio of features that have a p-value below certain thresholds corresponding to different levels
of significance. From the results it would seem that MobileNet and the NASNet variants may perform worse
than the rest of the networks, and VGG16 and VGG19 may follow the same trend. At the very least it seems
reasonable to bring each network with us in the next stages. We could perhaps discard MobileNet, but it
could be useful to keep as a sort of benchmark. We also don't know how many features need to be different
in order to make classification possible, and while the ratios for MobileNet are low, there's still
$0.227\cdot 2048 ~ 465$ different features, which is actually more features than MobileNetV2 and the Inception 
variants.
\begin{tabular}{l|c|r|r|r}
    \hline
     Network           &   features &   ratio $p < 0.01$ &   ratio $p < 0.005$ &   ratio $p < 0.001$ \\
    \hline
     DenseNet121       &            512 &         1.00  &          1.00   &          1.00   \\
     DenseNet169       &            512 &         1.00  &          1.00   &          1.00   \\
     DenseNet201       &            512 &         1.00  &          1.00   &          1.00   \\
     InceptionResNetV2 &            320 &         0.969 &          0.969  &          0.966  \\
     InceptionV3       &            320 &         0.981 &          0.981  &          0.981  \\
     MobileNet         &           2048 &         0.227 &          0.227  &          0.227  \\
     MobileNetV2       &            384 &         1.00  &          1.00   &          1.00   \\
     NASNetLarge       &           2058 &         0.510 &          0.510  &          0.510  \\
     NASNetMobile      &            539 &         0.510 &          0.510  &          0.510  \\
     ResNet50          &           4096 &         1.00  &          1.00   &          1.00   \\
     VGG16             &            512 &         0.633 &          0.633  &          0.627  \\
     VGG19             &            512 &         0.652 &          0.652  &          0.652  \\
     Xception          &           3200 &         1.00  &          1.00   &          1.00   \\
    \hline
    \caption{Comparison of feature distributions for single events and double events for all networks.
    The dataset is balanced, containing 100000 of each event type.}
    \label{tab:features}
\end{tabular}

\begin{tabular}{lrrrr}
    \hline
     Network           &   features &   ratio $p < 0.01$ &   ratio $p < 0.005$ &   ratio $p < 0.001$ \\
    \hline
     DenseNet121       &            512 &         1.00  &          1.00  &          1.00  \\
     DenseNet169       &            512 &         0.998 &          0.998 &          0.998 \\
     DenseNet201       &            512 &         1.00  &          1.00  &          1.00  \\
     InceptionResNetV2 &            320 &         0.969 &          0.969 &          0.966 \\
     InceptionV3       &            320 &         0.972 &          0.969 &          0.962 \\
     MobileNet         &           2048 &         0.222 &          0.221 &          0.219 \\
     MobileNetV2       &            384 &         1.00  &          1.00  &          1.00  \\
     NASNetLarge       &           2058 &         0.481 &          0.474 &          0.457 \\
     NASNetMobile      &            539 &         0.458 &          0.445 &          0.419 \\
     ResNet50          &           4096 &         1.00  &          0.999 &          0.999 \\
     VGG16             &            512 &         0.602 &          0.599 &          0.598 \\
     VGG19             &            512 &         0.643 &          0.643 &          0.639 \\
     Xception          &           3200 &         1.00  &          1.00  &          0.999 \\
    \hline
    \caption{Comparison of feature distributions for single events and double events with relative distance
	between events $< 3mm$. The dataset contain 100000 single events and 11006 double events.}
    \label{tab:features-close}

\end{tabular}
\section{Classification}
%\appendix
%\bibliographystyle{unsrt}
%\bibliography{bibliography}
\end{document}
