\section{Theory}
\subsection{Normalization}
\subsubsection{Pixel values}
To reduce the risk of exploding and/or vanishing gradients, pixel values are 
commonly normalized to $[0, 1]$ or to $[-1, 1]$ (with zero-mean). 
Here we have opted for a min-max scaling to the $[0, 1]$ interval.
For a set of images the minimum and maximum pixel values across all images
in the set are the ones used in calculations. This preserves differences in
intensities between images in the set, and also the shape of the intensity 
distribution. Min-max scaling is calculated as
\begin{equation}
    \text{scaled image} = \f{\text{image} - \mu_{image}}{I_{max} - I_{min}},
\end{equation}
where $I_{max}$ and $I_{min}$ refer to the maximum and minimum pixel intensity,
and $\mu_{image}$ is the mean pixel intensity for the set of images.

\subsection{Distances}
The data provides position of origin in units of pixels, such that coordinates
$x,y \in [0,16]$. We normalize these to the interval $[0,1]$ by dividing all
distances by 16.

\subsection{Metrics}
How do we evaluate the performance of a model? For regression the most common
metrics are the \textbf{Mean Absolute Error}, 
\textbf{Mean Squared Error}, \textbf{Root Mean Squared Error}, and the \textbf{R2 score}.

\subsubsection{Mean Absolute Error (MAE)}
The MAE is calculated by taking the mean of the absolute value of residuals.
Residuals are obtained by subtracting the predicted values ($\hat{y}$) from the 
corresponding true values ($y$).
The MAE is then defined as
\begin{equation}\label{metrics:mae}
    MAE = \f{1}{n}\sum_{i=0}^{n-1}\abs{y_i - \hat{y}_i}
\end{equation}

\subsubsection{Mean Squared Error}
The MSE is the mean of the squared residuals. Since the residuals are squared
before the mean is taken, large errors are weighted higher than small errors.
It is defined as
\begin{equation}\label{metrics:mse}
    \f{1}{n}\sum_{i=0}^{n-1}(y_i - \hat{y_i}^{2},
\end{equation}

\subsubsection{Root Mean Squared Error (RMSE)}
Taking the square root of the MSE (\ref{metrics:mse}) yields the RMSE.
This quantity is similar to the MAE, but weighs larger errors heavier
for the same reason the MSE does.
\begin{equation}
    RMSE = \sqrt{MSE}
\end{equation}
The RMSE and MSE both tell you something about the magnitude of the errors
in predictions, but they are useful in different situations. If some large
errors aren't a problem, MAE can be better, but if those large errors are
highly undesireable RMSE may give you better insight intoodel performance.

\subsubsection{$R^2$ Score}
The $R^2$ score can be seen as a measure of how much of the variance in the
predicted values ($\hat{y}$) is explained by the variance in the input ($y$).
Let the mean of the predicted values be given by
\begin{equation}\label{metrics:mean}
    \bar{y} =\f{1}{n}\sum_{i=0}^{n-1}\hat{y}_{i}.
\end{equation}
Using three sums of squares - the total sum of squares,
\begin{equation}
    SS_{tot} = \sum_{i}(y_i - \bar{y})^{2},
\end{equation}
the regression sum of squares,
\begin{equation}
    SS_{reg} = \sum_{i}(\hat{y}_i - \bar{y})^{2},
\end{equation}
and the residual sum of squares
\begin{equation}
    SS_{res} = \sum_{i}(y_i - \hat{y}_i)^{2},
\end{equation}
the most general definition of the R2 score is as
\begin{equation}
    R^2 \equiv 1 - \f{SS_{res}}{SS_{tot}}
\end{equation}

If your models performs with a score of $R^2 = 0.49$, then $49\%$ of the
variance in the predicted values can be explained by the variance in the input.
The remaining $51\%$ is unaccounted for. The maximum score is $R^2 = 1.0$,
in which the model predicts perfectly.


