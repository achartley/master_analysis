\section{Introduction}
A short summary of current work on using pre-trained neural networks to classify
single and double events in scintillator data.

\section{What are pre-trained networks and how do we use them?}
As the name suggests a pre-trained network is a network that has already
been trained, usually on very large amounts of data over a significant amount
of time. This is the case for the networks explored in this work. The networks
have all been trained on the "ImageNet" database and are on the cutting edge in
computer vision. They have also been analyzed extensively, increasing the amount
of insight in the learning process.

But we're not interested in classifying ImageNet data, we have our own data,
so how can we use the power of these networks for our own purposes?
The answer is \textbf{feature extraction}. When feeding an image forward
through the trained network, the various filters (or "kernels") react to different features
of the image, like edges or lines, general shapes etc. What we end up with on the other side
of the network is a "feature representation" of the input image. Hopefully, an input image
of say a cat and a car will produce significantly different feature representations, allowing
for classification. In our case we desire that the feature representations for single and
double events are different. This is the first thing we test when looking at which networks
might work for us.
